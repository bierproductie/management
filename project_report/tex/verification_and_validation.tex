\section{Verification \& Validation}
This section describes the verification and validation of the developed MES.\\

Verification ensures the software quality, design, architecture and the like by
checking if the software has been built, according to the plan. Validation is
testing if the developed system meets the brewery's requirements found in Table
\ref{table:Requirements}.

\subsection{Verification}
The source code is compared to the results of the architecture and design
sections to verify the developed system. \\

When comparing, the focus is on ensuring that the system complies with the
design principle, separation of concerns. The database should follow the
specified EER diagram, the dashboard should comply with the design pattern MVC,
and the OPC-UA client should communicate via HTTP. \\

The database can store relevant data, and the data corresponds with the tables
specified in the EER diagram. The dashboard follows the MVC pattern, as the
Model controls the data-related logic, the View controls the user interface
logic, and the Controller controls the business logic. The OPC-UA and the REST
API are separate subsystems that communicate through the HTTP protocol. 


\subsection{Validation}
The validation tests consists of producing every product type at different
machine speeds with the same number of products to produce and gather the
production data. These tests make it possible to calculate the OEE for every
speed and then calculate the error function for the product types.\\

The validations below are concluded during the tests and list the features and
the requirements they fulfil. They also mention features that extend the basic
need of the requirements. These do not matter as much, as the ones on the list
are the primary focus. They, however, are a good addition for the overall
usability.

\subsubsection{Control production} 
The system allows control of the production line. The production manager can
start, stop, clear, reset, and abort the production through the dashboard. This
functionality fulfils requirements R01-R02 in the list of requirements.

\subsubsection{Data processing}
The system monitors batch data from the production line. The batch data is
stored in the database and kept track of, using the batch id. This functionality
allows the data to be accessed in the history page on the dashboard after it has
run, thus fulfilling requirements R03-R06.

\subsubsection{Data presentation}
Having access to past batches and their data allows the dashboard to provide a
visualisation of batch reports and live data: fulfilling requirement R07, R08,
and R9.

\subsubsection{Optimisation}
Based on past and present data, the production line can be optimised. By
calculating OEE, estimating error function and finding the optimal production
speed, the brewery can optimise their production line. These features fulfil
requirements R10-R12 completing the list of requirements. \\

By looking at the implemented features, the system fulfils the requirements from
the Overall Requirements Specification.
