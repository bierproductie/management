\section{Requirements}

\subsection{Overall Requirements Specification}

\subsection{Selected Detailed Requirements}

\subsubsection{Functional \& Non-Functional Requirements}

\subsubsection{The Physical Setup (The Brewery Machine)}

The machine is able to produce a different kind of beers depending on the recipe
used. The beer is produced in batches which have their own batch id. When 
producing a batch the quantity, speed and batch id can be configured by an 
external system. When a batch is done the quality inspection system measures 
the quality for future use. These values are all accessible in the machines 
SCADA system that monitor and supervises both the machines PLC system and the 
internal sensors PLC system as well as the quality inspection system.

\myparagraph{Sensors}
To ensure the most optimal environment the machine is equipped with three 
sensors that measure while production is happening. This data is useful because 
it gives insight into the internal environment of the machine while producing 
products. Currently, it measures temperature, humidity and vibration. On that 
note is important to remember the machine itself is not able to collect these 
data and are collected by an external system.

\myparagraph{Quality Inspection System}
In order to make it possible to calculate an error function, the system is 
equipped with a sophisticated quality inspection system. This system is key to 
optimising the production by giving data about the different configurations 
error functions so that they can be configured otherwise to maximise production.


\subsubsection{The Simulator}


\subsection{Use Cases}

\subsubsection{Actor List}

\subsubsection{Detailed Use Cases}
\textit{From project description}

\subsubsection{Use Case Diagram}
