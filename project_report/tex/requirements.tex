\section{Requirements}

\subsection{Overall Requirements Specification}

\subsection{Selected Detailed Requirements}

\subsubsection{Functional \& Non-Functional Requirements}
In table \ref{table:sup_requirements} you can see a description of Functional and
Non-Functional requirements. 
FURPS+ is a model for classifying functional and non-functional requirements.
The acronym stands for functionality, usability, reliability, performance, and supportability.
The '+' sign stands for design constraints, implementation requirements,
interface requirements, and physical requirements.


\begin{table}[ht]
    \begin{tabularx}{\textwidth}{|>{\RaggedRight}p{3.3cm}|>{\RaggedRight}p{0.6cm}|>{\RaggedRight}X|}
        \hline
        \textbf{FURPS+}  & \textbf{\#} & \textbf{Demands} \\
        \hline
        Functionality  	& S01 & Improve the beer machine by increasing quantity while maintaining quality \\
        \hline
        Usability      	& S02 & Documentation on usage of the REST API \\
        \hline
        Reliability    	& S03 & On server reboot, the application will automatically restart \\
        \hline
        Performance    	& S04 & Max response time (API: 400 ms) \\
        \hline
        Supportability 	& S05 & Minimum browser versions (JavaScript version 6)\\
        \hline
        Design constraints 	& S06 & Na \\
        \hline
        \multirow{5}{100}{Implementation requirements} & S08 & Should be controlled via MES\\
        \cline{2-3}
                & S08 & MES should be able to keep track of Batches\\
        \cline{2-3}
                & S09 & Monitor production (Live data)\\
        \cline{2-3}
                & S10 & Estimate error function\\
        \cline{2-3}
                & S11 & Optimal production speed\\
        \hline
        \multirow{14}{100}{Interface requirements } & S12 & Show OEE \\
        \cline{2-3}
                & S13 & Show Batch Report that include:
            \begin{itemize}
                \item Batch ID
                \item Product type
                \item Amount of products (total, defect and acceptable)
                \item Amount of time used in the different states
                \item Logging of temperature over the production time
                \item Logging of humidity over the production time
            \end{itemize} \\
        \cline{2-3}
            & S14 & Visualisation \\
        \hline
        Physical requirements & S14 & The group should work with the beer 
        production machine provied by SDU \\
        \hline
    \end{tabularx}
    \caption{Supplementary Requirements} 
    \label{table:sup_requirements}
\end{table} 

\subsubsection{The Physical Setup (The Brewery Machine)}

The machine can produce different kinds of beer depending on the recipe
used. The beer is produced in batches which have their own batch id. When 
producing a batch the recipe, quantity, speed and batch id can be configured by 
an external system. When a batch has been produced the quality inspection system 
measures the quality for future use. These values are accessible in the machines 
SCADA system. The SCADA system monitors and supervises both the machines PLC 
system and the internal sensors PLC system as well as the quality inspection 
system.

\myparagraph{Sensors}
To ensure the most optimal environment beer production the 
machine is equipped with three sensors that measure the environment while 
production is happening. This data is useful as it gives insight into the 
internal environment of the machine while producing products. Currently, it 
measures temperature, humidity and vibration. The machine itself is not able to 
collect these data and are collected by an external system.

\myparagraph{Quality Inspection System}
To make it possible to calculate an error function, the system is 
equipped with a sophisticated quality inspection system. This system is key to 
optimising the production by giving data about the different error functions of 
the configurations so that they can be configured otherwise to maximise 
production. Each produced product is inspected and if the quality is low enough
the product will fail. When a batch is finished the QIS will output the amount 
of failed and passed products. The result is then used to calculate the error
function.


\subsubsection{The Simulator}
The group is going to use the simulator software to test the software during the
development cycle.
It is important to note that the simulator is not a replacement of the machine
since there is only so much randomness and correctness you can get from a 
simulator. 
It is still very important to have the simulator, when making prototypes and 
performing multiple tests on it, so any regressions won't be pushed to 
the production system(the beer machine).

\subsection{Use Cases}

\subsubsection{Actor List}

\subsubsection{Detailed Use Cases}
\textit{From project description}

\subsubsection{Use Case Diagram}
