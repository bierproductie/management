\section{Requirements}

\subsection{Overall Requirements Specification}

\subsection{Selected Detailed Requirements}

\subsubsection{Functional \& Non-Functional Requirements}

\subsubsection{The Physical Setup (The Brewery Machine)}

The machine can produce different kinds of beer depending on the recipe
used. The beer is produced in batches which have their own batch id. When 
producing a batch the recipe, quantity, speed and batch id can be configured by 
an external system. When a batch has been produced the quality inspection system 
measures the quality for future use. These values are accessible in the machines 
SCADA system. The SCADA system monitors and supervises both the machines PLC 
system and the internal sensors PLC system as well as the quality inspection 
system.

\myparagraph{Sensors}
To ensure the most optimal environment beer production the 
machine is equipped with three sensors that measure the environment while 
production is happening. This data is useful as it gives insight into the 
internal environment of the machine while producing products. Currently, it 
measures temperature, humidity and vibration. The machine itself is not able to 
collect these data and are collected by an external system.

\myparagraph{Quality Inspection System}
To order to make it possible to calculate an error function, the system is 
equipped with a sophisticated quality inspection system. This system is key to 
optimising the production by giving data about the different error functions of 
the configurations so that they can be configured otherwise to maximise 
production. Each produced product is inspected and if the quality is low enough
the product will fail. When a batch is finished the QIS will output the amount 
of failed and passed products. The result is then used to calculate the error
function.


\subsubsection{The Simulator}


\subsection{Use Cases}

\subsubsection{Actor List}

\subsubsection{Detailed Use Cases}
\textit{From project description}

\subsubsection{Use Case Diagram}
