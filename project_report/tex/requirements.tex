\section{Requirements}

\subsection{Overall Requirements Specification}

\subsection{Selected Detailed Requirements}

\subsubsection{Functional \& Non-Functional Requirements}
In table \ref{table:Sup_requirements} you can see a description of Functional and
Non-Functional requirements. 
FURPS+ is a model for classifying functional and non-functional requirements.
The acronym stands for functionality, usability, reliability, performance, and supportability.
The '+' sign stands for design constraints, implementation requirements,
interface requirements, and physical requirements.


\begin{table}[ht]
    \begin{tabularx}{\textwidth}{|>{\RaggedRight}p{3.3cm}|>{\RaggedRight}p{0.6cm}|>{\RaggedRight}X|}
        \hline
        \textbf{FURPS+}  & \textbf{\#} & \textbf{Demands} \\
        \hline
        Functionality  	& S01 & Improve the beer machine by increasing quantity while maintaining quality \\
        \hline
        Usability      	& S02 & Documentation on usage of the REST API \\
        \hline
        Reliability    	& S03 & On server reboot, the application will automatically restart \\
        \hline
        Performance    	& S04 & Max response time (API: 400 ms) \\
        \hline
        Supportability 	& S05 & Minimum browser versions (JavaScript version 6)\\
        \hline
        design constraints 	& S06 & Na \\
        \hline
        \multirow{5}{100}{implementation requirements} & S08 & Should be controlled via MES\\
        \cline{2-3}
                & S08 & MES should be able to keep track of Batches\\
        \cline{2-3}
                & S09 & Monitor production (Live data)\\
        \cline{2-3}
                & S10 & Estimate error function\\
        \cline{2-3}
                & S11 & Optimal production speed\\
        \hline
        \multirow{14}{100}{Interface requirements } & S12 & Show OEE \\
        \cline{2-3}
                & S13 & Show Batch Report that include:
            \begin{itemize}
                \item Batch ID
                \item Product type
                \item Amount of products (total, defect and acceptable)
                \item Amount of time used in the different states
                \item Logging of temperature over the production time
                \item Logging of humidity over the production time
            \end{itemize} \\
        \cline{2-3}
            & S14 & Visualization \\
        \hline
        Physical requirements & S14 & The group should work with the beer 
        production machine provied by SDU \\
        \hline
    \end{tabularx}
    \caption{Supplementary Requirements} 
    \label{table:Sup_requirements}
\end{table} 

\subsubsection{The Physical Setup (The Brewery Machine)}

\subsubsection{The Simulator}


\subsection{Use Cases}

\subsubsection{Actor List}

\subsubsection{Detailed Use Cases}
\textit{From project description}

\subsubsection{Use Case Diagram}
