\section{Requirements}

\subsection{Overall Requirements Specification}
\subsubsection{Problem Statement}
In the table \ref{table:problem-statement}, the finished problem, the problem statement and related questions are listed.
\begin{table}[ht]
    \begin{tabularx}{\textwidth}{|>{\RaggedRight}p{4cm}|>{\RaggedRight}X|}
        \hline
        \textbf{Problem} & The current production line is not effecient enough to keep up with the demand of the beer, while still maintaining a quality product\\
        \hline
        \textbf{Problem Statement} & How to control and optimise the brewing machine, to maximise the production of high quality beer\\
        \hline
        \textbf{Related questions} & 
            \begin{itemize}
                \item How can we optimise the production?
                \item How can we utilise calculus and linear algebra to provide a meaningful overview of the production line, based on statistics?
                \item How can we create a web based frontend for the MES?
                \item How can we separate the different aspects of the system (separation of concerns)
            \end{itemize}
        \\ 
        \hline
    \end{tabularx}
    \caption{Problem statement showcase} 
    \label{table:problem-statement}
\end{table} 

\subsubsection{Summary of requirements}
The group's proposed solution will adhere to the requirements given by the brewery Refslevbæk Bryghus A/S.

The manufacturing execution system, MES, must be able to control the brewery’s production.
It must be able to start and stop the production line,
as well as monitor the production and collect data from the production line.
The data must be stored for further analysis.
The MES must be able to keep track of the batches that the new machine is producing,
as well as collect various data from the machine that is associated with the current batch number.
After a finished batch production, the MES must be able to produce a batch report.
The report must contain the following.

\begin{itemize}
    \item This Batch ID
    \item Product type
    \item Amount of products (total, defect and acceptable)
    \item Amount of time used in the different states
    \item Logging of temperature over the production time
    \item Logging of humidity over the production time
\end{itemize}

The MES/SCADA (Supervisory control and data acquisition) system must be able to monitor the production and display live relevant data from the machine.
The documentation of the system must contain an illustration that defines the different components in the setup, in relation to the ISA88\^{}1213 Part 1 Physical Hierarchy model.
The system must have a visualisation that can be accessed and used to display production data.
The system must be able to collect the necessary data from the machine and calculate the overall equipment effectiveness, OEE\^{}131516, of the machine. The OEE must be available to be displayed by the system.
The system must be able to estimate the error function associated with the different products.
The system must be able to find the optimal production speed for each product type, based on an error simulation and the appertaining graph upon which the error simulation is built.

\subsubsection{List of requirements}
Below is a list of the above requirements. These requirements have been
prioritised using the MoSCoW method, where M is for Must have, S is for
Should have, C is for could have, and W is for Won't have. 

\begin{table}[H]
    \begin{tabularx}{\textwidth}{|>{\RaggedRight}p{1cm}|>{\RaggedRight}p{4cm}|>{\RaggedRight}X|>{\RaggedRight}p{1cm}|}
        \hline
        \textbf{ID} & \textbf{Name} & \textbf{Description} & \textbf{Prio} \\
        \hline
        R01 & Control production line & Control the brewery's production & M \\
        \hline
        R02 & Control production line & Start/stop production line & M \\
        \hline
        R03 & Monitor production & Monitor data from the production line & M \\
        \hline
        R04 & Monitor production & Store the collected data for further analysis & M \\
        \hline
        R05 & Administer batches & Keep track of produced batches (batch ID) & M \\
        \hline
        R06 & Store batch info & Collect various data associated with current batch number from the machine & M \\
        \hline
        R07 & Batch report & Produce a batch report (PDF/dashboard style format) & M \\
        \hline
        R08 & Live data & Monitor and display live relevant data from the machine & M \\
        \hline
        R09 & Documentation & Documentation must contain an illustration that defines the different components in the setup in relation to the ISA88\^{}1213 Part 1 Physical Hierarchy model & M \\
        \hline
        R10 & Visualisation & Visualisation that can be accessed and used to display the production data & M \\
        \hline
        R11 & OEE & Collect necessary data from the machine and calculate the OEE. OEE must be available to be displayed by the system & M \\
        \hline
        R12 & Estimate error function & Estimate the error function associated with the products & S \\
        \hline
        R13 & Optimal Production speed & Estimate the optimal production speed for each product type & M \\
        \hline
    \end{tabularx}
    \caption{List of requirements} 
    \label{table:Requirements}
\end{table}

\subsection{Selected Detailed Requirements}

\subsubsection{Functional \& Non-Functional Requirements}

\subsubsection{The Physical Setup (The Brewery Machine)}

\subsubsection{The Simulator}


\subsection{Use Cases}

\subsubsection{Actor List}

\subsubsection{Detailed Use Cases}
\textit{From project description}

\subsubsection{Use Case Diagram}
