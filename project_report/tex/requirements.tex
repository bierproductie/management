\section{Requirements}

\subsection{Overall Requirements Specification}

\subsection{Selected Detailed Requirements}

\subsubsection{Functional \& Non-Functional Requirements}

\subsubsection{The Physical Setup (The Brewery Machine)}

\subsubsection{The Simulator}


\subsection{Use Cases}

% \begin{table}[ht]
%    \begin{tabularx}{\textwidth}{|>{\RaggedRight}p{3cm}|>{\RaggedRight}X|>{\RaggedRight}p{2.5cm}|}
%     \hline
%     \textbf{Use Case ID}   & \textbf{Name}                                 & \textbf{Definition}\\ \hline
%     UC01                   & Start production line                         & User \\ \hline
%     UC02                   & Stop production line                          & User \\ \hline
%     UC03                   & Monitor and store data from production line   & SCADA \\ \hline
%     UC04                   & Monitor and display relevant data live        & SCADA \\ \hline
%     UC05                   & Create batch report                           & MES \\ \hline
%     UC06                   & Calculate OEE                                 & MES \\ \hline
%     UC07                   & Estimate the error function                   & MES \\ \hline
%     UC08                   & Find optimal production speed                 & MES \\ \hline
%     \end{tabularx}
%     \caption{Use Cases}
%     \label{table:use_cases}
%     \end{table}

\subsubsection{Actor List}

\begin{table}[ht]
     \begin{tabularx}{\textwidth}{|>{\RaggedRight}p{2.5cm}|>{\RaggedRight}p{8cm}|>{\RaggedRight}X|}
     \hline
     \textbf{Actor} 				& \textbf{Description}                                                                                                              & \textbf{Goal} \\ \hline
     \multirow{2}{*}{SCADA (p)}     & The SCADA system is responsible for saving data as well as acting as a communication link between the MES and the Hardware.       & 	\begin{itemize} 
     																																											\item Collect and save data
     																																											\item Communicate between hardware and MES 
     																																									 	\end{itemize} \\ \hline

     \multirow{2}{*}{User (p)}      & The user represents the worker of the machine. The user will use the GUI to control the machine.                                  & 	\begin{itemize}
     																																											\item Control machine
     																																											\item Start new batches
     																																										\end{itemize} \\ \hline
    \end{tabularx}
    \caption{Actor list}
    \label{table:actor_list}
\end{table}

\subsubsection{Detailed Use Cases}
\textit{From project description section 3.2}

% 3.2.1 start production
\begin{table}[ht]
    \begin{tabularx}{\textwidth}{|>{\RaggedRight}X|}
        \hline
        \textbf{ID:} UC01  \\
        \hline
        \textbf{Primary actor:} The MES \\
        \hline
        \textbf{Secondary actor:} Beer production machine \\
        \hline
        \textbf{Short description:} The MES must be able to start the brewery's
        production \\
        \hline
        \textbf{Pre conditions:} The beer production machine needs to be off \\
        \hline
        \textbf{Main flow:} \\
        	1. This use case starts when an actor wants to start the beer
        	production machine. \\
        	2. The actor chooses what kind of beer to be produced. \\
        	3. The actor presses the start button. \\
		\hline
        \textbf{post conditions:} The beer production machine is turned on \\
        \hline
        \textbf{Alternative flow:} \\
        	Step 3: If there is not enough ingredients, the actor receives an
        	error message. \\
        \hline
    \end{tabularx}
    \caption{Production Control: Start production}
    \label{table:usecase_productionStart}
\end{table}

% 3.2.1 stop production
\begin{table}[ht]
    \begin{tabularx}{\textwidth}{|>{\RaggedRight}X|}
        \hline
        \textbf{ID:} UC02  \\
        \hline
        \textbf{Primary actor:} The MES \\
        \hline
        \textbf{Secondary actor:} Beer production machine \\
        \hline
        \textbf{Short description:} The MES must be able to stop the brewery's
        production \\
        \hline
        \textbf{Pre conditions:} The beer production machine needs to be on \\
        \hline
        \textbf{Main flow:} \\
        	1. This use case starts when an actor wants to shut down the beer
        	production machine. \\
			2. The actor presses the stop button \\
		\hline
        \textbf{post conditions:} The beer production machine is turned off \\
        \hline
        \textbf{Alternative flow:} \\
        \hline
    \end{tabularx}
    \caption{Production Control: Stop production} 
    \label{table:usecase_productionStop}
\end{table}

% 3.2.1 monitor and store production data
\begin{table}[ht]
    \begin{tabularx}{\textwidth}{|>{\RaggedRight}X|}
        \hline
        \textbf{ID:} UC03  \\
        \hline
        \textbf{Primary actor:} The MES \\
        \hline
        \textbf{Secondary actor:} Beer production machine \\
        \hline
        \textbf{Short description:} The MES must be able to monitor and store
        the data from the production line. \\
        \hline
        \textbf{Pre conditions:} The production needs to be on and producing
        beer. \\
        \hline
        \textbf{Main flow:} \\
        	1. This use case starts when the beer production machine is turned
        	on and starts producing beer. \\
			2. The MES collects data from the beer production machine. \\
			3. The MES stores the collected data. \\
		\hline
        \textbf{post conditions:} Data from the beer production machine has been
        stored. \\
        \hline
        \textbf{Alternative flow:} \\
        \hline
    \end{tabularx}
    \caption{Production Control: Monitor and store data} 
    \label{table:usecase_productionStoreData}
\end{table}

% 3.2.3 production monitoring
\begin{table}[ht]
    \begin{tabularx}{\textwidth}{|>{\RaggedRight}X|}
        \hline
        \textbf{ID:} UC04  \\
        \hline
        \textbf{Primary actor:} The MES \\
        \hline
        \textbf{Secondary actor:} Beer production machine \\
        \hline
        \textbf{Short description:} The MES must be able to monitor the
        production and display live relevant data. \\
        \hline
        \textbf{Pre conditions:} The production needs to be on and producing
        beer. \\
        \hline
        \textbf{Main flow:} \\
        	1. This use case starts when the beer production machine is turned
        	on and starts producing beer. \\
			2. The MES collects data from the beer production machine. \\
			3. The MES displays the relevant data. \\
		\hline
        \textbf{post conditions:} Data from the beer production machine has been
        displayed. \\
        \hline
        \textbf{Alternative flow:} \\
        \hline
    \end{tabularx}
    \caption{Production Monitoring: Display live data}
    \label{table:usecase_displayLiveRelevantData}
\end{table}

% 3.2.4 batch report
\begin{table}[ht]
    \begin{tabularx}{\textwidth}{|>{\RaggedRight}X|}
        \hline
        \textbf{ID:} UC05  \\
        \hline
        \textbf{Primary actor:} The MES \\
        \hline
        \textbf{Secondary actor:} The beer production machine \\
        \hline
        \textbf{Short description:} The MES must be able to produce a batch
        report of the produced batch. \\
        \hline
        \textbf{Pre conditions:} The beer production machine needs to have
        produced a batch. \\
        \hline
        \textbf{Main flow:} \\
        	1. This use case starts when the beer production machine has
        	finished producing a batch of beer. \\
			2. The MES retrieves the data it stored during production. \\
			3. The MES produces a batch report with the retrieved data. \\
			4. The MES exports the batch report to an easily readable format \\
		\hline
        \textbf{post conditions:} A batch report has been produced. \\
        \hline
        \textbf{Alternative flow:} \\
        \hline
    \end{tabularx}
    \caption{Batch report: producing a batch report}
    \label{table:usecase_batchReport}
\end{table}

% 3.2.7 OEE
\begin{table}[ht]
    \begin{tabularx}{\textwidth}{|>{\RaggedRight}X|}
        \hline
        \textbf{ID:} UC06  \\
        \hline
        \textbf{Primary actor:} The MES \\
        \hline
        \textbf{Secondary actor:} The beer production machine \\
        \hline
        \textbf{Short description:} The MES needs to collect the necessary data
        from the machine and calculate the OEE, and store it for
        further references. \\
        \hline
        \textbf{Pre conditions:} The beer production machine needs to have
        produced a batch. \\
        \hline
        \textbf{Main flow:} \\
        	1. This use case starts when the beer production machine has
        	finished producing a batch of beer. \\
			2. The MES collects the necessary data from the beer production
			machine \\
			3. The MES calculates the OEE. \\
			4. The MES stores the calculated OEE for further reference. \\ 
		\hline
        \textbf{post conditions:} The OEE has been calculated and is ready to
        be displayed. \\
        \hline
        \textbf{Alternative flow:} \\
        \hline
    \end{tabularx}
    \caption{OEE: Calculating the efficiency of the machine}
    \label{table:usecase_oee}
\end{table}

% 3.2.8 estimate error function
\begin{table}[ht]
    \begin{tabularx}{\textwidth}{|>{\RaggedRight}X|}
        \hline
        \textbf{ID:} UC07  \\
        \hline
        \textbf{Primary actor:} The MES \\
        \hline
        \textbf{Secondary actor:} The beer production machine \\
        \hline
        \textbf{Short description:} The MES needs to estimate the error function
        associated with the products. \\
        \hline
        \textbf{Pre conditions:} The beer production machine needs to have
        produced a batch. \\
        \hline
        \textbf{Main flow:} \\
        	1. This use case starts when the beer production machine has
        	finished producing a batch of beer. \\
			2. The MES collects the necessary data from the beer production
			machine \\
			3. The MES estimates the error function. \\
			4. The MES stores the error function for the produced batch for
			further reference. \\
		\hline
        \textbf{post conditions:} The error function has been estimated. \\
        \hline
        \textbf{Alternative flow:} \\
        \hline
    \end{tabularx}
    \caption{Error function: Estimate error function}
    \label{table:usecase_errorFunction}
\end{table}

% 3.2.9 optimal production speed
\begin{table}[ht]
    \begin{tabularx}{\textwidth}{|>{\RaggedRight}X|}
        \hline
        \textbf{ID:} UC08  \\
        \hline
        \textbf{Primary actor:} The MES \\
        \hline
        \textbf{Secondary actor:} The beer production machine \\
        \hline
        \textbf{Short description:} The optimal production speed must be
        estimated for each product type. \\
        \hline
        \textbf{Pre conditions:} The beer production machine needs to have
        produced a batch. \\
        \hline
        \textbf{Main flow:} \\
        	1. This use case starts when the beer production machine has
        	finished producing a batch of beer. \\
			2. The MES collects the necessary data from the beer production
			machine \\
			3. The MES calculates the optimal production speed for the produced
			product. \\
			4. The MES stores the calculated production speed for further
			reference. \\ 
		\hline
        \textbf{post conditions:} The optimal production speed for the batch
        product type has been estimated. \\
        \hline
        \textbf{Alternative flow:} \\
        \hline
    \end{tabularx}
    \caption{Optimal production speed: Calculating the efficiency of the machine}
    \label{table:usecase_productionSpeed}
\end{table}

\subsubsection{Use Case Diagram}
