\section{Introduction}

% Motivation from project proposal
As students, this semester project gives the group great learning experience as
well as a good idea of how such a project would progress. During the project,
the group will learn how to create and set up an Open Platform Communications
Unified Architecture, OPC-UA, client, and how to make it communicate directly
to a physical machine. The group will learn to host a web server, from which the
machine can be controlled, and access it as a website. The group will also have
the chance to learn and use a new scripting language, JavaScript. The gained
experience is the main motivation of this project.

% Problem formulering
\subsection{Problem Statement}
In the table \ref{table:problem-statement}, the finished problem, the problem statement and related questions are listed.
\begin{table}[ht]
    \begin{tabularx}{\textwidth}{|>{\RaggedRight}p{4cm}|>{\RaggedRight}X|}
        \hline
        \textbf{Problem} & The current production line is not effecient enough to keep up with the demand of the beer, while still maintaining a quality product\\
        \hline
        \textbf{Problem Statement} & How to control and optimise the brewing machine, to maximise the production of high quality beer\\
        \hline
        \textbf{Related questions} & 
            \begin{itemize}
                \item How can we optimise the production?
                \item How can we utilise calculus and linear algebra to provide a meaningful overview of the production line, based on statistics?
                \item How can we create a web based frontend for the MES?
                \item How can we separate the different aspects of the system (separation of concerns)
            \end{itemize}
        \hline
    \end{tabularx}
    \caption{Problem statement showcase} 
    \label{table:problem-statement}
\end{table} 

% Overview of project
\subsection{Overview of project}
This project aims to improve the beer machine by increasing quantity while 
maintaining quality. To accomplish this the project will offer an interactive 
web interface for monitoring, controlling and adjusting the brewery machine. 
This is done by having the web interface interact with a REST API that connects 
to a client. This client controls the machine through an OPC-UA server 
connection and stores relevant data in a database.
