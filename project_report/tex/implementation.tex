\section{Implementation}
From the design section, it was clarified how the system was to be designed, and
this section will focus on how the subsystems are implemented, using that
knowledge.

\subsection{Dashboard}
The user interacts with the dashboard, seen in Figure \ref{figure:dashboard}, and
through it, the user can see what the machine is producing and information about
the production. The user can also interact with the dashboard, as they can start
and stop the machine and press buttons to see live data represented in graphs.
The user can start a new batch for the machine to produce, as well as see
the batch history, to get an overview of produced batches and its data.\\

\begin{figure}[ht]
	\centering
	\includegraphics[width=1\linewidth]{images/dashboard.png}
	\caption{Dashboard}
	\label{figure:dashboard}
\end{figure}

The dashboard is implemented using web-based technology, so it can run on
multiple platforms without installing any software, apart from a modern browser
that is running JavaScript. It is implemented using HTML, CSS and JavaScript,
and it uses two separate libraries for making graphs and getting icons for the
buttons.

\subsection{REST API}
The REST API is responsible for providing the dashboard with live data, and the
ability to start new batches from the dashboard. The API is able to provide
live data due to the OPC-UA client posting it to the API as soon as a property
changes. The data is then kept in the PostgreSQL database. The API is developed
in Python3 via the popular FastAPI framework. For interacting with the database,
it is using the async ORM Gino. The source code structure is inspired by best
practices in MVC, with separations between the different layers (router,
service, schemas and database models (ORM)). This ensures that components can
easily be swapped out, and thereby makes it easier to maintain in the long run.\\

Another part of implementation is the development setup via a continuous
integration pipeline. In the pipeline it is verified if the code is linted
properly, as well as verifying if unit tests and system tests are passing. If
one of these steps fails, the pull request is not allowed to be merged to master.
When new branches are merged to master, the last part of the pipeline is being
run which builds a new docker image to a private registry, which is then deployed
automatically to \url{https://api.bierproductie.nymann.dev/docs}. This can be
verified via the version on the docs page matching the latest commit hash.


\subsection{Database}
The database is implemented with PostgreSQL, which is a relational database.
The PostgreSQL software is reliable and has been used for mange years in the
industry. The table implementation in the database is derived from the EER
Diagram, see Figure \ref{figure:eer_diagram} for reference. All communication to
the database in the production setup is routed through the REST API. This way,
data quality and easier error handling is ensured. From a security
standpoint, this is superior since it is possible to make an internal docker
network. The network is only shared with the REST API, which ensures that none
other than the REST API can connect to it. The database consists of the
following tables \texttt{data\_over\_time}, \texttt{recipes}, \texttt{batches} and
\texttt{alembic\_version}. The \texttt{data\_over\_time} table is where live data
is posted to by the client. To improve performance, this table is partitioned on a
monthly basis based on the measurement timestamp of each row. Given the system
design, an interesting point is the delay between the machine
measurement timestamp, and when the data is inserted in the database. This is
implemented via a function that returns the current timestamp and is triggered
when data is inserted on the \texttt{data\_over\_time} table.


\subsection{OPC-UA Client}
The OPC-UA client handles the data from the machine and sends it to the API. It 
is written in Java and uses the Milo library to create a client that connects to 
the machines OPC-UA-server. The OPC-UA client is able to subscribe to 
information on the server and send it to the API that stores it in the database.
The client also controls the machine while it collects data, as it creates and
runs batches it gets from the dashboard through the API. The client sends live
data to the dashboard via the API using subscriptions. These subscriptions send
data to the API every time a change appears, and the API then forwards it to the
dashboard. The client creates six subscriptions when started, and they monitor
temperature, vibration, humidity, machine-state, current defective products and
current processed products. \\

The data is also stored in an object called Batch in the client. The Batch class
has properties corresponding to the values the API uses as well as logic to
process the data. This logic includes OEE calculation and JSON exportation. In
the start of a batch, when the user has configured the desired production, the
data is sent to the client to create a Batch object. The Batch object is then
run with the BatchHandler class, and while the batch is being produced, the
subscriptions will add data to the object. When done the Batch object will run
through its logic and finally export itself as a JSON so it can be sent to the
API. 


\subsection{Overall Equipment Effectiveness}
When a batch is complete, and the data is stored in the batch class, the
calculation of the OEE takes place.\\

The planned production time can not be estimated for this project, as too many
variables are unknown (shift lengths, the time it takes to refill ingredients,
the time it takes to do maintenance work etc.). Therefore, it has been decided
to use the time it takes to produce a batch. The calculation can be seen
below.

\[PlannedProductionTime = \left(\frac{AmountToProduce}{MachineSpeed}\right)\cdot60\]\\

Where the amount to produce divided by the machine speed gives the production
time in minutes, as the machine speed is products per minute. This result can be
multiplied by 60 to give the production time in seconds. The maximum machine
speed is needed to find the ideal cycle time. 1 divided by this multiplied by
60 then results in the theoretical minimum time it takes to produce one unit.

\[IdealCycleTime = \frac{1}{maxMachineSpeed}\cdot60\]\\

When these have been calculated, it is possible to calculate the OEE using
Equation \ref{eq:OEE}.

\[OEE = \left(\frac{AcceptedProducts\cdot{IdealCycleTime}}{PlannedProductionTime}\right)\cdot100\]\\

This result gives the overall equipment effectiveness in percentage, which is
used to evaluate if the production can be optimised further. It should be noted
that the result from this calculation is only indicative, as neither quality,
performance or availability can be assessed from this.

To give an example on how a calculation with real values would look like, take
the values: amount to produce = 100, maximum machine speed = 300, batch machine
speed = 200, accepted products = 97.

\[PlannedProductionTime = \left(\frac{100}{200}\right)\cdot60 = 30\]

\[IdealCycleTime = \frac{1}{300}\cdot60=0.2\]

\[OEE = \left(\frac{97\cdot0,2}{30}\right)\cdot100 = 64,67\%\]\\

The OEE for this specific batch would be 64,67\%, which means that the
production line can be optimised further if needed.

\subsection{Error Function}
The error function associated with each product type is to be estimated. For
this, several batches should be produced to have some data to do the
calculations with. The variables used to estimate the error function is the
amount of products to produce, the machine speed and the amount of rejected
products. These values are used to make an expression, that resembles the error
function. The data used can be found in Appendix \ref{app:batch_data}.

\begin{figure}[ht]
	\centering
	\includegraphics[width=1\linewidth]{images/errorfunction/pilsner.png}
	\caption{Estimated Error Function for Pilsner}
	\label{figure:ef_pilsner}
\end{figure}

The estimated error function seen in Figure \ref{figure:ef_pilsner} gives an
idea of when the beer production machine starts to produce many rejected
products. All the estimated error functions for the products the machine can
produce can be seen in Table \ref{table:eef}, and the graphs can be found in
Appendix \ref{app:error_function_graphs}.

\begin{table}[ht]
    \captionof{table}{Estimated Error Functions}
     \begin{tabularx}{\textwidth}{|>{\RaggedRight}p{3cm}|>{\RaggedRight}X|>{\RaggedRight}p{3cm}|}
     \hline
     \textbf{Product type} & \textbf{Estimated Error Function} & \textbf{Accuracy} \\
     \hline
     Pilsner & \(y = 0,0003x^3 - 0,0291x^2 + 0,95x - 7,3838\) & \(R^2 = 0,9905\) \\
     \hline
     Wheat & \(y = 5E-05x^3 - 0,0077x^2 + 1,3163x - 3,6317\) & \(R^2 = 0,9983\) \\
     \hline
     IPA & \(y = -0,0001x^3 + 0,0548x^2 - 3,7529x + 72\) & \(R^2 = 0,9981\) \\
     \hline
     Stout & \(y = -0,0002x^3 + 0,0319x^2 - 1,551x + 67,071\) & \(R^2 = 0,9331\) \\
     \hline
     Ale & \(y = -2E-05x^3 + 0,0099x^2 - 0,5615x + 9,1857\) & \(R^2 = 1\) \\
     \hline
     Alcohol Free & \(y = 3E-05x^3 - 0,0021x^2 + 0,4464x + 10\) & \(R^2 = 0,9935\) \\
     \hline
    \end{tabularx}
    \label{table:eef}
\end{table}

\subsection{Optimal Production Speed}
To find the optimal production speed, the same data as when estimating the error
function is used. As the error simulation in the machine is a function of the
normalised machine speed and the selected product, the normalised machine speed
must be calculated. This is done by taking 100 over the maximum machine speed.
This, multiplied by the chosen machine speed, will then give the percentage of
the maximum machine speed, which is used when plotting the graph for the optimal
production speed. The normalised machine speed will be the \(x\)-values and the
calculated OEE will be the \(y\)-values. An example can be seen in Figure
\ref{figure:ops_pilsner}, where the graph for the optimal production speed for
the product type pilsner can be seen.

\begin{figure}[ht]
	\centering
	\includegraphics[width=1\linewidth]{images/ops/pilsner.png}
	\caption{Optimal Production Speed for Pilsner}
	\label{figure:ops_pilsner}
\end{figure}

This plot is generated using Microsoft Excel. When the OEE for each tested
machine speed is plotted, a third-degree polynomial trend line is applied, and an
expression for the function generated is given. For the pilsner, the function is
given by:

\[y = -0,0003x^3+0,0334x^2-0,1057x+9,0202\]


With an \(R^2\) of \(0,9905\) which is used to see how close to the data points
the polynomial is (the closer to 1 the more accurate). To find the maximum of
the function, the derivative is calculated:

\[y' = -\frac{9x^2-668x+1057}{10000}\]

To find the maximum, this equation should be solved for \(x\) when \(y=0\). This
gives \(72,61\), which is the optimal machine speed for the pilsner product type,
measured in percentage of the maximum machine speed. To make sure the calculated
\(x\)-value is a local maximum at \(72,61\), the second derivative is calculated:

\[y'' = -\frac{18x-668}{10000}\]

Then, this equation is solved for \(y\) when \(x=72,61\), which gives a value of
\(-0,06\). Since the result is less than 0, it can be confirmed that it is
a maximum.\\

This is done for all product types, and the results can be seen in Table
\ref{table:ops}.


\begin{table}[ht]
    \captionof{table}{Optimal Production Speed}
     \begin{tabularx}{\textwidth}{|>{\RaggedRight}p{3cm}|>{\RaggedRight}X|>{\RaggedRight}p{6cm}|}
     \hline
     \textbf{Product type} & \textbf{Maximum} & \textbf{Optimal Production Speed}\\
     \hline
     Pilsner & 72,61\% of 600 & 435,63 products/minute \\
     \hline
     Wheat & 51,58\% of 300 & 154,75 products/minute \\
     \hline
     IPA & 62,55\% of 150 & 93,83 products/minute \\
     \hline
     Stout & The equation has no solution for \(x\) in ${\rm I\!R}$ & 200 products/minute (max) \\
     \hline
     Ale & 91,82\% of 100 & 91,82 products/minute \\
     \hline
     Alcohol Free & 76,35\% of 125 & 95,44 products/minute \\
     \hline
    \end{tabularx}
    \label{table:ops}
\end{table}


The reason for the result of the stout is that the function has no local maxima
in the range (0 to 200). Instead, it keeps increasing, as seen in the graph for the
stout in Appendix \ref{app:ops_graphs}. This means that the faster the machine speed,
the better, which is why the optimal production speed is the maximal machine
speed.
