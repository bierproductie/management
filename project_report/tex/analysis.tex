\section{Analysis}

\subsection{Use Case analysis}

\subsubsection{Class Candidates}
In order to find potential class candidates, every noun of the detailed Use Cases are found.
These are potential candidates, and can be sorted to avoid duplicates and candidates that 
won’t be turned into classes. Naturally, every potential class for the entire system will 
not be found, as this only reflects use cases. A potential class candidate such as MES 
(where Start and Stop functionality would otherwise be implemented) will not be reduced to 
a single class and is therefore not added to the list of class candidates.

The final list of classes, as well as a description of them, can be seen in table (x).

\begin{table}[ht]
    \begin{tabularx}{\textwidth}{|>{\RaggedRight}p{4cm}|>{\RaggedRight}p{6cm}|>{\RaggedRight}X|}
    \hline
    \textbf{Class Candidate} & \textbf{Attributes}                                                                                                     & \textbf{Definition}                                                                    \\ \hline
    Batch                    & Id, type, product\_amount (total, defect, acceptable), amount (time), state (current, history), OEE, production\_speed, & A batch refers to a specific batch of products the brewery has made                    \\ \hline
    Product                  & Id, type, Ingredients,                                                                                                  & Product refers to the different options of beer to be produced                         \\ \hline
    Ingredient               & Name, id                                                                                                                & An ingredient refers to a specific ingredient. Products contain a list of ingredients. \\ \hline
    \end{tabularx}
    \caption{Potential class candidates}
    \label{table:class_candidates}
    \end{table}

\subsubsection{UML Analysis Diagram}

\subsection{Use Case Realisation}

\subsubsection{Sequence Diagrams}

\subsubsection{Operation Contracts}

\subsubsection{Updated UML Class Diagram}
