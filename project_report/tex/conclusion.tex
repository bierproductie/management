\section{Conclusion}
The group has successfully developed an MES, consisting of a dashboard, an
OPC-UA client and a REST API. The dashboard, developed using HTML, CSS, and
JavaScript, allows the user to control the production line. The OPC-UA client
connects to the beer production machine, making it possible to collect data and
control the machine by making the machine's endpoints available to the
dashboard. The client is developed in Java. The REST API, written in Python,
acts a translator between the two aforementioned subsystems as well as the
database. This allows a flow of data between the user, using the dashboard, and
the machine, using the OPC-UA client. By developing three subsystems, the MES 
complies with the design principle, separation of concerns. \\

By using calculus and linear algebra, the group has been able to provide essential
data to the brewery regarding the optimisation of their production line. This
includes estimating the error function, found in Table \ref{table:eef}, and
finding the optimal production speed, found in Table \ref{table:ops}, for each
product type, based on the overall equipment effectiveness. \\

If the brewery chooses to use the developed MES, they will be able to control
and optimise the brewing process to maximise the production of high-quality beer.