\section{Architecture}
From the analysis of the of the project it is clear that an OPC-UA
client is needed, as this should connect to the beer production machine. This
client should in some way store the collected data from the production machine,
which means that some kind of database should be a part of the MES. As the data
only consists of simple data types, integers, strings, etc., it is decided
to use a relational database. This way, data from different batches can easily
be stored in such a way that the relations between the data can be kept as
needed. By having a database, the subsystems do not have to store data
locally, thus making the same data available for other parts of the system,
securing data consistency. \\

It was discussed whether to develop a desktop client functioning as a dashboard
to control the production machine or if it should be a web client. A web client 
is chosen, as it is a more modern solution, it is more accessible than a desktop
client, as it can run in all supported web browsers and does not need to be
installed. If the company then chooses to change hardware, the dashboard is
still functional. \\
 
To make the data from the database available to all subsystems in the MES, 
it was decided to develop a REST API. This API acts as a translator between the
subsystems in the MES, simplifying the data management, e.g. when the
dashboard needs to show the data from previous batches. Instead of having the
dashboard communicate directly with the database, a layer is added, the REST
API, which both adds security and handles the queries uniformly. The database
then sends a query result to the REST API which then translates the SQL to JSON
for the dashboard to read and display to the user. \\

An overview of the MES to be developed can be seen in Figure
\ref{figure:architucture_diagram}.

\begin{figure}[ht]
	\centering 
	\includegraphics[width=1\textwidth]{images/diagrams/architecture_diagram.png}
	\caption{Software Architecture Diagram}
	\label{figure:architucture_diagram} 
\end{figure}
