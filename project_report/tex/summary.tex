\section{Summary}
The purpose of this project has been to develop an Manufacturing Execution
System for the brewery Refslevbæk Bryghus A/S, to optimise the production line
so the brewery can keep up with the demand for their products. The brewery
recently bought a new brewing machine, and a simulator of this has been made
available to the group. From analysing the given project description, the group
has narrowed the problem down to the following:\\

\textit{"How to apply an MES to control and optimise the brewing process at
Refslevbæk Bryghus A/S, to maximise the production of high-quality beer."}\\

The report contains an overview over the development of the MES, from initial
requirements analysis through architecture, design, and implementation resulting
in verification and validation of the developed system. Based on these sections
the developed system can be evaluated and concluded upon, asserting if the
group has successfully solved the problem.\\

In the end, the group has managed to develop an MES consisting of a dashboard,
an API and an OPC-UA Client. The dashboard acts as a control panel, from which a
production can be started and stopped. Real-time data from the production is
displayed, and graphs can be viewed to give an overview over the production.
Furthermore, a history page can be accessed, containing data regarding past
batches. The data used to provide this functionality is stored in a database.
The stored data is also used to optimise the production line by calculating the
Overall Equipment Effectiveness and finding the optimal production speed for
each product type. The data is made available by the OPC-UA client, which
connects the beer production machine, and then posts the data to the API. The
API then inserts the data in the database, and makes it available for the
dashboard to consume.