\section{Summary}
The purpose of this project has been to develop an MES for the brewery
Refslevbæk Bryghus A/S, in order to optimize the production line. The brewery
has recently bought a new brewing machine to keep up with the demand.
Based on the initial requirements analysis, the group has narrowed the problem
down to the following:\\

\textit{"How to apply an MES to control and optimise the brewing process at
Refslevbæk Bryghus A/S, to maximise the production of high-quiality beer."}\\

During the development, the group has made use of the framework Scrum, to
organise the group work. Every week, a meeting has been held to plan ahead in
the development process. These issues are assigned to a two week sprint. To
manage this, the management solution, ZenHub, has been used alongside GitHub.
For each task, an issue has been created and a group member has been assigned.\\

The group has managed to develop an MES consisting of a frontend
website, an API and an OPC Client. The website acts as a control panel, from
which a beer batch can be started. The website displays relevant data in real
time, which can be interacted with. From the frontend, a history page with 
information, regarding past batches, can be accessed. The data is used for batch
reports and further analysis, and is stored in a database. The system also
calculates the Overall Equiptment Effectiveness (OEE) to furhter optimise the
production line.