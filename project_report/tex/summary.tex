\section{Summary}
The purpose of this project has been to develop an MES for the brewery
Refslevbæk Bryghus A/S, to optimise the production line so the brewery
can keep up with the demand. The brewery recently bought a new brewing machine,
and a simulator of this has been made available to the group. An initial
requirements analysis of the project case has been made and based on this; the
group has narrowed the problem down to the following:\\

\textit{"How to apply an MES to control and optimise the brewing process at
Refslevbæk Bryghus A/S, to maximise the production of high-quality beer."}\\

During the development phase of the Unified Process, the group has made use of
the management framework Scrum, to organise the group work. A meeting has been
held every week, to plan the next step in the development process as well as
make issues for the planned tasks. These issues are assigned to a two-week
sprint. To manage this, the management solution, ZenHub, has been used alongside
GitHub. For each task, an issue has been created, and a group member has been
assigned. \\

The group has managed to develop an MES consisting of a dashboard, an API and an
OPC-UA Client. The dashboard acts as a control panel, from which a beer batch
can be started and stopped, the machine can be cleared and reset. Real-time data
is displayed, which the user can interact with in order to see detailed graphs.
A history page can be accessed, containing data regarding past batches. The data
is stored in a database and is used for batch reports and further analysis. The
stored data is also used to optimise the production line, e.g. by
calculating the Overall Equipment Effectiveness (OEE). The REST API is used as a
translator between the dashboard and the OPC-UA client.