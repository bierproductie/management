\section{Evaluation}
The evaluation section describes how the reflections of the goals, set for the
project, meets the customer demands, as well as how the development process has
been.


\subsection{Evaluation from the customer point of view}
The group has managed to meet the requirements, set in the initial requirements
analysis at the beginning of the project. The created system is able to control
the production line, using the dashboard, as well as collect, display and
store relevant data in a database. The data is used for further analysis, such
as calculating the OEE and optimal machine speed, as well as estimating the
error function.
The design of the dashboard makes it easy to see what is going on the front page.
Every item on the front page is there; nothing is hiding in a sub-menu.
Besides starting the beer machine, the start button would need to be pressed
to redirect the user to a start-batch page where a recipe, speed and the amount
to produce can be chosen. The start-batch page is not that convenient, because
the page  does not put in the recommended values for the different recipes, but
it does show the recommended values for the recipe.
In order to see past batches, the history page should be navigated to. This page
is easily accessible from the front page by pressing on the button left
side menu bar. Here, past batches can be selected to inspect further. The button
layout is the same as the front page, so there should be no learning curve there
to check out the previous batches. Overall the group thinks it is a well-working
system for people not familiar with the project because it does comply with the
standard UI formalities, and there is almost no learning curve to the dashboard.


\subsection{Development process}
At the beginning of the development process, the group has held a meeting
about how the projects should be structured. Building on the experience from
last semester where Unified Process and Scrum were used, the group
decided to use both UP and Scrum again. Every Friday, a Scrum meeting is held,
to show and discuss what each group member has accomplished. Each task is
assigned to a two-week sprint, which is planned at the appropriate Scrum
meeting.
During 'matching of expectations', the group agreed upon focusing on the Minimum
Viable Product, meaning not implementing parts outside of the semester project's
scope. This has been a more realistic goal in comparison to last semester.
The use of Scrum has given a great overview of the project and has made it 
easier to plan ahead. However, looking back, the group should have scheduled 
more in the beginning. Near the end, the group has had long work hours.

The project has been divided into the four phases of Unified Process; Inception,
Elaboration, Construction, and Transition. Each phase has been a fixed boundary
to plan around, making it easier to schedule sprints, tasks, and issues. The
group could have worked more efficiently in the initial phases of Unified
Process, enabling the group to begin the Construction phase sooner.
