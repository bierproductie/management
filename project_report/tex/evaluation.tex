\section{Evaluation}
The evaluation section describes how the reflections of the goals, set for the
project, meets the customer demands, as well as how the development process has
been.


\subsection{Evaluation from the customer side}
The group has managed to meet the requirements, set in the initial requirements
analysis, in the beginning of the project. The created system is able to control
the production line, using the dashboard, and is able to collect, display and
store relevant data in a database. The data is used for further analysis, such
as calculating the OEE and optimal machine speed, as well as estimating the
error function.
The design of the dashboard, makes it easy to see what is going on the front-page.
Every item on the front-page is there nothing is hiding down under a sub menu,
beside starting the beer machine, you would need to press the start button,
that will then take you to a start-page where you can chose a recipe, speed and 
the amount to produce. The start-page is not that convenient, because the page 
does not put in the recommend values for the different recipes, but it does
show the recommend values for the recipe.
For seeing other older batches you will need to navigate to a different page
called history.html. You can navigate to this page easy from the front-page by
pressing on the button left side menu bar that takes you to the history page.
Where you will be able to chose a batch to inspect.
The button layout is the same as the front-page, so there should be no learning
curve there to check out the old batches.
Overall the group thinks it is a good system for people not familiar with the
project, because it does comply with the standard UI formalities and there is
almost no learning curve to the dashboard.


\subsection{Development process}
In the beginning of the development process, the group has held as meeting
about how the projects should be structured. Building on the experience from
last semester where Unified Process and Scrum was used, the group
decided to use both UP and Scrum again. Every Friday, a Scrum meeting is held,
to show and discuss what each group member has accomplished. Each task is
assigned into a two week sprint, which is planned at the appropriate Scrum
meeting.
During 'matching of expectations', the group agreed upon focusing on the Minimum
Viable Product, meaning not implementing parts outside of the semester project's
scope. This has been a more realistic goal in comparison to last semester.
The use of Scrum has giving a great overview of the project, and has made it 
easier to plan ahead. However, looking back, the group should have scheduled 
more in beginning. Near the end, the group has had longer work hours.

The project has been divided into the four phases Unified Process; Inception,
Elaboration, Construction, and Transition. Each phase has been a fixed boundary
to plan around, making it easier to schedule, sprints, tasks, and issues.
Unlike the previous semester, no inception document was created. Instead, the
initial requirements analysis was incorporated directly into the final report.
The initial phases of Unified Process could have been more efficient, enabling
the group to begin the Construction phase sooner.
