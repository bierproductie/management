\section{Design}

\subsection{Database Design}
The project should have a database to store data. 
The data that should be stored in the database is something that 
resembles a batch. In the figure \ref{figure:eer_diagram_batch} it can be seen
what and how the data should be stored in the database.

Uses for the database is:
For example, if a person would like to 
know what happened when the beer was produced,
like what was the highest temperature during the production, did that have a
any negative effect on the beer.
How many products were defect out of the total amount produced. 

\begin{figure}[ht]
\centering 
\includegraphics[width=0.8\linewidth]{images/eer_diagrams/database_EER_batch.png}
\caption{IR diagram for batch} 
\label{figure:eer_diagram_batch}
\end{figure}

\subsection{Dashboard design}
A very important aspect the group has had to concider, is which design pattern
to model the dashboard after. A design pattern is a form of template, which in
this case, dictates how the different classes communicate. A relevant design
pattern is the MVC (Model - View - Controller). This pattern separates the
data-related logic (model), the UI logic (View) and the business logic 
(controller) into different layers, and thereby helps keep separation of
concerns. The controller manages input from the user, and forwards it to either
the model, to retrieve or store data, or the view, to present data to the user.

The group has decided to use the MVC design pattern, thus making it possible to
work on the different aspects of the system (GUI, database, logic) separately.
To ensure that these different parts of the system can communicate, a
'contract', should be created.

A visualisation of the MVC pattern can be seen in figure \ref{figure:MVC_model}

\begin{figure}[ht]
    \centering
    \includegraphics[scale=0.15]{images/MVC_model.png}
    \caption{MVC model}
    \label{figure:MVC_model}
\end{figure}

\subsection{Subsystem Design}
\subsubsection{OPC-UA Client}
The group was in doubt about the system being a monolith application containing 
the REST-API and OPC-UA Client in one program or split into two programs.

Choosing to split the program into two separate entities aligns with the 
separation of concerns principle fx. implementing authentication at a later 
stage(in the API) wouldn't affect the core logic of the OPC-UA client. On the 
other hand choosing to do so would add some overhead at the start of the 
implementation phase. \\

Another thing to consider when splitting the application into two separate 
entities is how they should interface with each other. To accomplish this the 
group has two options: 

\begin{table}[ht]
    \begin{tabularx}{\textwidth}{|>{\RaggedRight}X|>{\RaggedRight}X|>{\RaggedRight}X|}
        \hline
        \textbf{Option} & \textbf{Pros} & \textbf{Cons} \\
        \hline
        Console Application & Simplest to implement & Connection to the OPC-UA 
        server is created for each request (Performance overhead)\\
        \hline
        HTTP server & Connection to OPC-UA server is made once and reused. HTTP
        naturally talks JSON which is ideal as an interface & Takes longer to 
        implement \\
        \hline
    \end{tabularx}
    \caption{Options}
    \label{someLabel}
\end{table}

The group evaluated that the split up benefits outweighed the negatives. This 
decision demands that the client uses HTTP communication to communicate with the
API over localhost. 

\myparagraph{Results}
\myworries{Write after implementation}