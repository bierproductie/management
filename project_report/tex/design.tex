\section{Design}
\subsection{Subsystem Design}
\subsubsection{OPC-UA Client}
The group was in doubt about the system being a monolith application containing 
the REST-API and OPC-UA Client in one program or split into two programs.

Choosing to split the program into two separate entities aligns with the 
separation of concerns principle fx. implementing authentication at a later 
stage(in the API) wouldn't affect the core logic of the OPC-UA client. On the 
other hand choosing to do so would add some overhead at the start of the 
implementation phase. \\

Another thing to consider when splitting the application into two separate 
entities is how they should interface with each other. To accomplish this the 
group has two options: 

\begin{table}[ht]
    \begin{tabularx}{\textwidth}{|>{\RaggedRight}X|>{\RaggedRight}X|>{\RaggedRight}X|}
        \hline
        \textbf{Option} & \textbf{Pros} & \textbf{Cons} \\
        \hline
        Console Application & Simplest to implement & Connection to the OPC-UA 
        server is created for each request (Performance overhead)\\
        \hline
        HTTP server & Connection to OPC-UA server is made once and reused. HTTP
        naturally talks JSON which is ideal as an interface & Takes longer to 
        implement \\
        \hline
    \end{tabularx}
    \caption{Options}
    \label{someLabel}
\end{table}

The group evaluated that the split up benefits outweighed the negatives. This 
decision demands that the client uses HTTP communication to communicate with the
API over localhost. 

\myparagraph{Results}
\myworries{Write after implementation}