\section{Design}
In this section, it will be described in detail how the different subsystems of
the MES, described in the previous sections, will be designed. The previous
section, architecture, acts as the blueprint for this part of the design phase.

\subsection{Database}
The database is used to store all data regarding batches. Based on the
Enhanced Entity-Relationship Diagram (EER) in Figure \ref{figure:eer_diagram},
the SQL files can be written.

\begin{figure}[ht]
\centering 
\includegraphics[width=0.8\linewidth]{images/eer_diagrams/database_EER_batch.png}
\caption{EER Diagram} 
\label{figure:eer_diagram}
\end{figure}

The database will have four main tables; batches, TempAndTime, VibrationAndTime and
HumidityAndTime. The batch table consists of all the data related to a
specific batch, such as id, product\_type, amount\_to\_produce, etc. The three
other tables will contain a time value and related data (temperature, vibration
and humidity).
Connecting the three tables to the batch table are "subscription tables".
These will link a batch to temperature, vibration, and humidity, respectively,
using the id.

\subsection{Dashboard}
A very important aspect the group has had to consider, is which design pattern
to model the dashboard after. A design pattern is a template, which in
this case, dictates how the different classes communicate. A relevant design
pattern is the MVC (Model - View - Controller). This pattern separates the
data-related logic (model), the UI logic (View) and the business logic 
(controller) into different layers, and thereby helps keeping separation of
concerns. The controller manages input from the user, and forwards it to either
the model, to retrieve or store data, or the view, to present data to the user.\\

The group has decided to use the MVC design pattern, thus making it possible to
work on the different aspects of the system (GUI, database, logic) separately. 
A visualisation of the MVC pattern can be seen in Figure \ref{figure:MVC_model}

\begin{figure}[H]
    \centering
    \includegraphics[scale=0.15]{images/MVC_model.png}
    \caption{MVC model}
    \label{figure:MVC_model}
\end{figure}

\subsection{OPC-UA Client}
The group discussed about the system being a monolith application containing 
the REST-API and OPC-UA client in one program or if it should be split into two
programs.\\

Choosing to split the program into two separate entities aligns with the 
separation of concerns principle, e.g. implementing authentication at a later 
stage (in the API) wouldn't affect the core logic of the OPC-UA client. On the 
other hand, choosing to do so would add some overhead at the start of the 
implementation phase. \\

Another thing to consider when splitting the application into two separate 
entities is how they should interface with each other. To accomplish this the 
group has two options: 

\begin{table}[ht]
    \captionof{table}{Development Options}
    \begin{tabularx}{\textwidth}{|>{\RaggedRight}X|>{\RaggedRight}X|>{\RaggedRight}X|}
        \hline
        \textbf{Option} & \textbf{Pros} & \textbf{Cons} \\
        \hline
        Console Application & Simplest to implement & Connection to the OPC-UA 
        server is created for each request (Performance overhead)\\
        \hline
        HTTP server & Connection to OPC-UA server is made once and reused. HTTP
        naturally talks JSON which is ideal as an interface & Takes longer to 
        implement \\
        \hline
    \end{tabularx}
    \label{someLabel}
\end{table}

The group evaluated that the split up benefits outweighed the negatives. This 
decision demands that the client uses HTTP communication to communicate with the
API over localhost. 
